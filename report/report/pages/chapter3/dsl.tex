\newpage

\section{Domain Specific Language} 
\label{sec:Domainspecificlanguage}

	A Domain Specific Language in many circles is generally described as a computer programming language with limited expressiveness,
	focused on a particular domain.  The best example of this description is the hypertext markup language (HTML) and
	of course everyone's favourite the structured query language(SQL).
	\newline
	\newline
	Domain specific languages are not a new concept.  
	The structured query language(SQL) was conceived 1986 \& the hypertext markup language (HTML) 1989.
	The choice for a Domain Specific Language (DSL) depends largely on what you are trying to achieve and whether this
	succinct language offers more efficient, less verbose and more flexibility; but even more so, is it better suited for the domain.
	Would it be efficient for people to do database queries in the Extensible Markup Language (XML)?   For those who are familiar with
	oracle's structured query language(SQL) command line interface, imagine using it with the Extensible Markup Language (XML).
	\newline
	\newline
	Moving quickly on from this eye brow raising statement, the choice for a Domain Specific Language in the solution was evident.
	Having experience in other solutions, such as INI files, Javascript Object Notation (JSON) and Extensible Markup Language (XML);
	these seemed rather verbose and restrictive to administrators who wished to define rules quickly and cleanly.
	Language cacophony is a common objection to taking on a Domain Specific Language (DSL).  However this argument is quickly avoided
	by the purpose of the Domain Specific Language (DSL).  The purpose of the Domain Specific Language (DSL) is to reduce the complexity of the statements
	used to achieve a domain specific goal.  Writing a web site in Java would be quiet a difficult and time consuming task; 
	this is why hypertext markup language (HTML) was conceived.
	\newline
	\newline
	As PERL was used as the client side implementation, this offered the perfect solution for text processing and the foundation for a Domain Specific Language (DSL).
	PERL, the practical extraction and reporting language was developed for text processing.  PERL has many tools for this activity, one of them being Filters.
	Filters or pre-processing of an input file, allowed for the formulation of this conceptual requirement.  As pre-processing can analyse the domain specific
	language, post-processing can also execute internal or embedded PERL statements, giving rise to the concept of a hybrid Domain Specific Language (DSL).
	\newline
	\newline
	Stepping back from this assertion for the moment, what is the difference between an ``internal'' and an ``external'' Domain Specific Language (DSL)?
	To be short and succinct in the definition of these two terms, an internal Domain Specific Language (DSL) is code that can be evaluated and executed
	without syntax directed translation taking place, to achieve the native language understood by the parser; while an external Domain Specific Language (DSL) is code
	that requires some sort of translation before it is understood by the parser.
	\newline
	\newline
	The following sections will identify these concepts in the implementation.
	
\newpage	

	\subsection{Requirements}
	
		\normalsize
		{
			\begin{enumerate}[itemsep=1pt,parsep=1pt]
				\item 	An extensible Domain Specific Language.
				\item 	Domain Specific Language grammar hooks.
				\item 	Hybrid Domain Specific Language.
				\begin{itemize}
					\item Internal characteristics, rules provided by modules.
					\item External characteristics, embedded PERL.
				\end{itemize}
			\end{enumerate} 		
		}
	
	\subsection{Design}
	
		\normalsize
		{
			The design of the Domain Specific Language (DSL) is based primarily on the extensibility of the client interpreter.  In previous
			sections we looked at how new parsing rules were injected into the interpreter.  These rules essentially define the Domain Specific Language (DSL).
			Therefore the language is not restricted by a document type definition (syntax and ordering rules).
			This means that the Domain Specific Language (DSL) constantly changes as interpreter modules are created and edited.
			Therefore the design of the Domain Specific Language (DSL) is the sole responsibility of the PERL module developers.
			\newline
			\newline
			Fig. \ref{fig:DSLHybrid} shows both examples of the internal \& external Domain Specific Language (DSL) statements.
			The internal Domain Specific Language (DSL) statements are identified between the \% symbols.
			An example of an external Domain Specific Language (DSL) statement, namely ``SERVICE\_CONTROL'' will be parsed by the rules
			provided by an interpreter module.
		}
		
		\vspace{2mm}
		\begin{figurehere}
			\inputminted[linenos=true,fontsize=\footnotesize,tabsize=2]{perl}{pages/chapter3/smippets/dsl2}
			\vspace{-5mm}
			\caption{DSL - Hybrid}
			\label{fig:DSLHybrid}
		\end{figurehere}	
		
		\vspace{2mm}
		\normalsize
		{
			Parsing of the internal statements is quiet simple and articulated in Fig. \ref{fig:InternalParsing},
			while the parsing of the external statements is shown in Fig. \ref{fig:ExternalParsing}
		}
		
		\vspace{2mm}
		\begin{figurehere}
			\inputminted[linenos=true,fontsize=\footnotesize,tabsize=2]{perl}{pages/chapter3/smippets/pe.txt}
			\vspace{-5mm}
			\caption{DSL - External Parsing}
			\label{fig:ExternalParsing}
		\end{figurehere}

\newpage		

		\vspace{4mm}
		\begin{figurehere}
			\inputminted[linenos=true,fontsize=\footnotesize,tabsize=2]{perl}{pages/chapter3/smippets/pi.txt}
			\vspace{-5mm}
			\caption{DSL - Internal Parsing}
			\label{fig:InternalParsing}
		\end{figurehere}			
	
	\subsection{Analysis \& Improvements}
	
		\normalsize
		{
			The hybrid characteristics of the Domain Specific Language (DSL) offers the greatest possible flexibility for administrators.
			By allowing the embedment of PERL into the Domain Specific Language (DSL), administrators who are waiting for new additional functionality to be provided 
			by me the developer, can work around any deficiencies discovered within the implementation.
		}
	
\newpage
	
