\chapter{Discussion and Conclusion}

	\normalsize
	{
		Providing a conclusion at this point is difficult considering the amount of material covered.
		Therefore in this section as part of the achievements, I will provide a simplified requirements mapping.
		Finally a critique and improvements points will be covered.
	}

	\section{Achievements}
	
	\normalsize
	{
		The following tables directly maps to the requirements in section \ref{sec:Scope} (scope).
	}
	
	\vspace{5mm}
	\large{\bfseries{Client Server Model}}
	\vspace{4mm}
	
	\begin{tablehere}
	
		\small{\begin{tabular}{ | p{5mm} | p{16mm} | p{100mm} | p{16mm} |}
		\hline
		\textbf{\#}							
		
		& \textbf{Achieved} 				& \textbf{Notes} 																									& \textbf{Supporting Figures}													\\ \hline
		1	 	 	 						& Yes					& 5 distributions tested for demo day				 										& \ref{fig:SSClient}															\\ \hline
		2	 	 	 						& Yes					& Modules can be pushed to clients 															& \ref{fig:SSServerControl}														\\ \hline
		3	 	 	 						& Yes					& Yes PERL was used																			& \ref{fig:SSModuleEditor}														\\ \hline
		4	 	 	 						& Yes					& Server tested to handle 6,000 clients														& \ref{fig:ServerCpuUsage}, \ref{fig:ServerQueueUsage} 
																																								\& \ref{fig:ServerNetworkUsage}													\\ \hline
		5	 	 	 						& Partially				& Extension points in the architectural design provides for the master slave paradigm		& \ref{fig:Transitionstatesmarshaling}											\\ \hline
		6	 	 	 						& Yes					& Support was catered for in state transitions points								 		& \ref{fig:Transitionstatesmarshaling}											\\ \hline
		7	 	 	 						& Yes					& Interpreted language used, deployable to multiple architectures							&																				\\ \hline
		8	 	 	 						& Yes					& Push and pull implemented																	& \ref{fig:SSServerControl}														\\ \hline
		
		\end{tabular}}
		
		\caption{Requirements Mapping - Client server model}
		\label{tab:RequirementsMappingClientservermodel}
		
	\end{tablehere}
	
	\vspace{5mm}
	\normalsize
	{
		The client server model as detailed in the development chapter is highly successfully. During the requirements capture it was deemed the implementation
		would need to be able to handle 100's of clients.   In the testing phase it was discovered that the architecture far exceeded this goal 
		and is able to handle thousands of clients.  
	}
	
	\vspace{5mm}
	\large{\bfseries{Domain Specific Language}}
	\vspace{4mm}
	
	\begin{tablehere}
	
		\small{\begin{tabular}{ | p{5mm} | p{16mm} | p{100mm} | p{16mm} |}
		\hline
		\textbf{\#}							& \textbf{Achieved} 	& \textbf{Notes} 																			& \textbf{Supporting Figures}													\\ \hline
		1	 	 	 						& Yes					& DSL is extensible																			& \ref{fig:ObserverDesignPattern} , \ref{fig:ClientInterpretor}					\\ \hline
		2	 	 	 						& Yes					& Client accepts new modules																& \ref{fig:ObserverDesignPattern} , \ref{fig:ClientInterpretor}					\\ \hline
		3	 	 	 						& Yes 					& DSL is an PERL filter script and allows embedded PERL										& \ref{fig:DSL}																	\\ \hline
		
		\end{tabular}}
		
		\caption{Requirements Mapping - Domain specific language}
		\label{tab:RequirementsMappingDomainspecificlanguage}
	
	\end{tablehere}
	
	\vspace{5mm}
	\normalsize
	{
		With the use of PERL filters the domain specific language exhibited both characteristics of an internal and external domain specific language.
		As the domain specific language is interpreted by the PERL interpreter, this allowed for the embedding of PERL, resulting in maximum flexibility.
	}
	
\newpage
	
	\vspace{5mm}
	\large{\bfseries{Directory Services Schema}}
	\vspace{4mm}
	
	\begin{tablehere}
	
		\small{\begin{tabular}{ | p{5mm} | p{16mm} | p{100mm} | p{16mm} |}
		\hline
		\textbf{\#}							& \textbf{Achieved} 	& \textbf{Notes} 																			& \ref{fig:dbdesignoverview}													\\ \hline
		1	 	 	 						& Yes					& Configurations represented in DB															& \ref{fig:dbdesignoverview}													\\ \hline
		2	 	 	 						& Yes					& Yes module provides translation															& \ref{fig:dbdesignoverview}													\\ \hline
		3	 	 	 						& Yes					& Yes organisational units represented														& \ref{fig:dbdesignoverview}													\\ \hline
		4	 	 	 						& Yes					& Client represented by GUID																& \ref{fig:dbdesignoverview}													\\ \hline
	
		\end{tabular}}
		
		\caption{Requirements Mapping - Directory services schema}
		\label{tab:RequirementsMappingDirectoryservicesschema}
		
	\end{tablehere}
	
	\vspace{5mm}
	\normalsize
	{
		The database schema although ending up be quiet simple, provided support for all the necessary requirements.
	}
	
	
	\vspace{5mm}
	\large{\bfseries{Administrator Front End}}
	\vspace{4mm}
	
	\begin{tablehere}
	
		\small{\begin{tabular}{ | p{5mm} | p{16mm} | p{100mm} | p{16mm} |}
		\hline
		\textbf{\#}							& \textbf{Achieved} 	& \textbf{Notes} 																			& \textbf{Supporting Figures}													\\ \hline		
		1	 	 	 						& Yes					& Ability to create organisational units													& \ref{fig:SSOuAdd}																\\ \hline
		2	 	 	 						& Yes					& Move organisational units																	& \ref{fig:SSOuMove}															\\ \hline
		3	 	 	 						& Yes					& Hierarchy of organisational units															& \ref{fig:SSouDragDRop}, \ref{fig:SSClient} \& \ref{fig:SSOuMove}				\\ \hline
		4	 	 	 						& Yes					& MUST provide a means to modify policies													& \ref{fig:SSPolicyEditor}														\\ \hline
		5	 	 	 						& Yes					& Generate new domain specific language														& \ref{fig:SSPolicyEditor}														\\ \hline
		6	 	 	 						& Yes					& Import existing policies in the form of the DSL											& \ref{fig:SSPolicyEditor}														\\ \hline
		7	 	 	 						& Yes					& Push updates to the client																& \ref{fig:SSServerControl}														\\ \hline

		\end{tabular}}
		
		\caption{Requirements Mapping - Administrator front end}
		\label{tab:RequirementsMappingAdministratorfrontend}	
		
	\end{tablehere}
	
	\vspace{5mm}
	\normalsize
	{
		All the requirements for the Administrator front end were achieved.  However the true achievement is the architecture.
		This architecture provides maximum flexibility and support for non functional requirements and resulting future improvements.
	}
					
	
	\section{Critique \& Improvements}
	
		\normalsize
		{
			There are numerous critiques of the software implementation.
			To begin with the implementation a primary critique point of the client, server architecture is the message passing.
			The messages as previously discussed are encapsulated in a Javascript Object Notation (JSON) envelope.  
			This envelope and the construction of its contents (key/value pairs) is hard coded in the implementation.  
			This does not offer out of band services to append to the (key/value) pairs.
			New services that require this functionality are therefore not catered for.
			Future improvements to the framework should provide the ability to append to this envelope.
			\newline
			\newline
			Patterns and supporting technologies such as the Common Object Request Broker Architecture (CORBA) should be considered
			to support object passing.
			\newline
			\newline
			A possible problem associated with the interpreter is the use of regular expressions.  Regular expressions
			although offering the best validation method for strings can become difficult to understand and maintain.
			Other languages such as Prolog provide the ability to define definite clause grammars more easily.
			For future improvements more research should be done to determine the best solution to defining rules
			in a more readable manner.
			\newline
			\newline
			Finally the last and most overlooked software concern of this project was the database requirement.
			Although the implementation catered for the requirements, further research needs to be done
			in order to better improve the representation schema for rules.  To provide the most flexibility during development
			the rules associated with an organisational unit were stored in a single field in the database.
			Although offering the best flexibility future improvement could be done to represent each line and its individual parts.
			Database internal constraints could then be used to further validate the data.
			\newline
		}
		
	\section{Conclusion}
	
		\normalsize
		{
			From the outset, the scope of this project was to design an extensible software framework and to provide a Domain Specific Language(DSL)
			as a means to abstract distribution specific knowledge.  These two points were successfully achieved in both the administrator front end and 
			the	client interpreter, leading to a uniform acknowledgement of quality, of the demonstrated product.	
			\newline
			\newline
		}
