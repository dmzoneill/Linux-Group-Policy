\section{Objectives}

	\normalsize
	{
		The following section depicts some high level artefacts for the framework under development:
		\newline
	}

	\large{\bfseries{Client Server Model}}
	
	\vspace{2mm}
	\normalsize
	{
		Centralised management being the theme of this report establishes the inception of the concept of a central authority and 
		therein a client server model.   One of the fundamental requirements of such an implementation is to provide a client server 
		model that can operate on a multitude of different systems with varying revisions of supporting packages.  
		\newline
		\newline
		For highest availability possible the client server should be implemented in language that runs on the majority 
		of Linux systems.  The client should be able to support updating as new functionality becomes available. This implies it 
		must be extensible, and as such a compiled language would be least preferable as there would be hardware architecture (Alpha, ARM, PPC, X86 x86\_64 \& SPARC)
		concerns in a large environment.   An interpreted language such as Perl supports these two constraints.
		\newline
	}
	
	\large{\bfseries{Administrator Front End}}	
	
	\vspace{2mm}
	\normalsize
	{
		The server or central authority must be controllable in some fashion.  As seen in Windows Active Directory, 
		a user interface is provided to accomplish this.  As the administrator interface provides the ability of managing the configuration 
		of this evolving environment, it must be extensible.  Therefore good software architecture concerns must be considered 
		when designing this interface. 
		\newline
		\newline
		In a large environment it is unlikely that the same policies will be applied to all the machines.  The management of a large 
		environment must provide a logical grouping of computers, allowing for the application of different polices to different groups.  
		As an example, a University would have logical distinction between student and staff computers, with different concerns and the 
		resulting policies.  This grouping of computers is generally referred to as organisational units.
		\newline
	}
		
	\large{\bfseries{Directory Services Schema}}		
	
	\vspace{2mm}
	\normalsize
	{
		The logical environmental representations or data must be stored somewhere for easy retrieval and modification.  
		A database typically would be used for such a purpose and the schema should be able to represent computer accounts 
		and be logically divided into organisational units.  Policies which will be applied to these organisational units must 
		also be stored in some fashion.
		\newline
	}
	
	\large{\bfseries{Domain Specific Language}}	
	
	\vspace{2mm}
	\normalsize
	{
		The domain specific language has to represent an abstraction of distribution specific tools and characteristics.  
		Therefore as the operating systems change and support new functionality, the domain specific language 
		must also support change. 
	}
