\vspace{-2mm}
\section{Overview Of The Final Year Project}
	
	\normalsize
	{	
		``Group policy'' a term more commonly associated with the ``Microsoft Windows Active Directory'' is defined as ``a set of rules
		that govern an environments user and computer accounts''.  Group policy provides a means of centralising the 
		management and therein the control of the configuration, of client operating systems and their features.  
		\newline
		\newline
		Centralised management in a Linux environment is seen as a far more difficult problem.  
		The environment, given the multitude of different distributions, all subscribing to their own 
		implementation philosophies has brought about almost infinite diversity.  This requires companies to 
		employ highly skilled technicians to manage these ever changing platforms.  
		These distributions or ``flavours'', seen to be highly different, however employ the same characteristics 
		in their fundamental implementation; only differing in the tools provided to control these characteristics.	
		\newline
		\newline		
		This presents a problem to administrators who wish to apply common rules to all of these operating systems from a central location.   
		A way to overcome this problem would be to provide an abstraction from these differing vendor tools.
		\newline		
		\newline
		Windows Active Directory Group Policy Object (GPO) editor has applied this concept.  Since Microsoft is the most successful company 
		in enterprise management, it is deemed that this approach provides the best, well understood solution to this common problem.  
		As each new Microsoft operating system is released it provides the ability to transform rules into operating system specific settings, 
		even though many of the tools may have changed dramatically.  
		\newline
		\newline
		The scope of this document therefore is to design an extensible composite framework that provides centralised
		management in a Linux environment.  A further objective is to realise an abatement of highly skilled technicians, that are 
		knowledgeable in all the varying flavours of Linux, by providing an abstraction in the form of a Domain Specific Language (DSL).				
	}

	\index{Domain Specific Language} 
	\index{Group Policy} 
	\index{Policies} 
	